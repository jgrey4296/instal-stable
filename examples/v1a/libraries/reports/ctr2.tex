% events = 10
% states = 10
% nstates = 6
% display width = 6
% states_by_row = [[5, 6, 7, 8, 9, 10]]

\begin{longtable}{@{}l@{}}
% start row=[5, 6, 7, 8, 9, 10] of [[5, 6, 7, 8, 9, 10]]

\resizebox{\textwidth}{!}{

\begin{tikzpicture}
[
start chain=trace going right,
% state chains for 5 through 10
start chain=state5 going down,
start chain=state6 going down,
start chain=state7 going down,
start chain=state8 going down,
start chain=state9 going down,
start chain=state10 going down,
node distance=1cm and 5.2cm
]
% start row=[5, 6, 7, 8, 9, 10], state=5
{{ [continue chain=trace]
\node[circle,draw,on chain=trace](i5) {$S_{5}$};
\draw[color=white](i5)+(180:5.6cm) --node[above]{}(i5);
\draw(i5)+(-3,0)node[rotate=90,anchor=south]{Answer set=1, source=Result/ctr2.res};
}
{ [continue chain=state5 going below]
\node [on chain=state5,below=of i5,rectangle,draw,inner frame sep=0pt] (s5) {
% instant 5
\begin{minipage}{5cm}\raggedright\everypar={\hangindent=1em\hangafter=1}
\textbf{inDebt(\allowbreak{}bob, book1\_a): pubA}\\
allowance(\allowbreak{}alice, 1): libB\\
allowance(\allowbreak{}alice, 2): libC\\
allowance(\allowbreak{}alice, 2): pubA\\
allowance(\allowbreak{}bob, 1): libC\\
allowance(\allowbreak{}bob, 2): libB\\
allowance(\allowbreak{}bob, 2): pubA\\
available(\allowbreak{}book1\_b): libB\\
available(\allowbreak{}book2\_a): pubA\\
available(\allowbreak{}book2\_b): libB\\
available(\allowbreak{}book2\_c): libC\\
available(\allowbreak{}book3\_a): pubA\\
available(\allowbreak{}book4\_a): pubA\\
borrowed(\allowbreak{}bob, book1\_a): pubA\\
borrowed(\allowbreak{}bob, book1\_c): libC\\
loans(\allowbreak{}alice, 0): libB\\
loans(\allowbreak{}alice, 0): libC\\
loans(\allowbreak{}alice, 0): pubA\\
loans(\allowbreak{}bob, 0): libB\\
loans(\allowbreak{}bob, 1): libC\\
loans(\allowbreak{}bob, 1): pubA\\
normalUser(\allowbreak{}alice): libB\\
normalUser(\allowbreak{}alice): libC\\
normalUser(\allowbreak{}alice): pubA\\
normalUser(\allowbreak{}bob): libB\\
normalUser(\allowbreak{}bob): pubA\\
\end{minipage}
};
} % end node and chain
\draw (i5) -- (s5);

}
% \pause % uncomment here to animate

% start row=[5, 6, 7, 8, 9, 10], state=6
{{ [continue chain=trace]
\node[circle,draw,on chain=trace](i6) {$S_{6}$};
% label for s_5 -- s_6
\draw[-latex,thin](i5) -- % t=6
node[above]{\begin{tabular}{>{\centering}m{5cm}}
\\
\em borrow(\allowbreak{}alice, book1\_c): libC\\
 borrow(\allowbreak{}alice, book1\_c): libB\\
 borrow(\allowbreak{}alice, book1\_c): pubA\\
 viol(\allowbreak{}borrow(\allowbreak{}alice, book1\_c)): libB\\
 intBorrow(\allowbreak{}alice, book1\_c): pubA\\
 due(\allowbreak{}bob, book1\_a): pubA\\
 issueFine(\allowbreak{}bob, book1\_a): pubA\\
 intBorrow(\allowbreak{}alice, book1\_c): libC
\end{tabular}}
(i6);
}
{ [continue chain=state6 going below]
\node [on chain=state6,below=of i6,rectangle,draw,inner frame sep=0pt] (s6) {
% instant 6
\begin{minipage}{5cm}\raggedright\everypar={\hangindent=1em\hangafter=1}
\textbf{borrowed(\allowbreak{}alice, book2\_c): libC}\\
\textbf{inDebt(\allowbreak{}bob, book1\_c): libC}\\
\textbf{loans(\allowbreak{}alice, 1): libC}\\
allowance(\allowbreak{}alice, 1): libB\\
allowance(\allowbreak{}alice, 2): libC\\
allowance(\allowbreak{}alice, 2): pubA\\
allowance(\allowbreak{}bob, 1): libC\\
allowance(\allowbreak{}bob, 2): libB\\
allowance(\allowbreak{}bob, 2): pubA\\
available(\allowbreak{}book1\_b): libB\\
available(\allowbreak{}book2\_a): pubA\\
available(\allowbreak{}book2\_b): libB\\
available(\allowbreak{}book2\_c): libC\\
available(\allowbreak{}book3\_a): pubA\\
available(\allowbreak{}book4\_a): pubA\\
borrowed(\allowbreak{}bob, book1\_a): pubA\\
borrowed(\allowbreak{}bob, book1\_c): libC\\
inDebt(\allowbreak{}bob, book1\_a): pubA\\
loans(\allowbreak{}alice, 0): libB\\
loans(\allowbreak{}alice, 0): pubA\\
loans(\allowbreak{}bob, 0): libB\\
loans(\allowbreak{}bob, 1): libC\\
loans(\allowbreak{}bob, 1): pubA\\
normalUser(\allowbreak{}alice): libB\\
normalUser(\allowbreak{}alice): libC\\
normalUser(\allowbreak{}alice): pubA\\
normalUser(\allowbreak{}bob): libB\\
normalUser(\allowbreak{}bob): pubA\\
\sout{loans(\allowbreak{}alice, 0): libC}\\
\end{minipage}
};
} % end node and chain
\draw (i6) -- (s6);

}
% \pause % uncomment here to animate

% start row=[5, 6, 7, 8, 9, 10], state=7
{{ [continue chain=trace]
\node[circle,draw,on chain=trace](i7) {$S_{7}$};
% label for s_6 -- s_7
\draw[-latex,thin](i6) -- % t=7
node[above]{\begin{tabular}{>{\centering}m{5cm}}
\\
\em borrow(\allowbreak{}alice, book2\_c): libC\\
 borrow(\allowbreak{}alice, book2\_c): libB\\
 borrow(\allowbreak{}alice, book2\_c): pubA\\
 viol(\allowbreak{}borrow(\allowbreak{}alice, book2\_c)): libB\\
 intBorrow(\allowbreak{}alice, book2\_c): pubA\\
 intBorrow(\allowbreak{}alice, book2\_c): libC\\
 due(\allowbreak{}bob, book1\_c): libC\\
 issueFine(\allowbreak{}bob, book1\_c): libC
\end{tabular}}
(i7);
}
{ [continue chain=state7 going below]
\node [on chain=state7,below=of i7,rectangle,draw,inner frame sep=0pt] (s7) {
% instant 7
\begin{minipage}{5cm}\raggedright\everypar={\hangindent=1em\hangafter=1}
\textbf{inDebt(\allowbreak{}bob, book1\_c): libB}\\
\textbf{loans(\allowbreak{}alice, 0): libC}\\
allowance(\allowbreak{}alice, 1): libB\\
allowance(\allowbreak{}alice, 2): libC\\
allowance(\allowbreak{}alice, 2): pubA\\
allowance(\allowbreak{}bob, 1): libC\\
allowance(\allowbreak{}bob, 2): libB\\
allowance(\allowbreak{}bob, 2): pubA\\
available(\allowbreak{}book1\_b): libB\\
available(\allowbreak{}book2\_a): pubA\\
available(\allowbreak{}book2\_b): libB\\
available(\allowbreak{}book3\_a): pubA\\
available(\allowbreak{}book4\_a): pubA\\
borrowed(\allowbreak{}bob, book1\_a): pubA\\
borrowed(\allowbreak{}bob, book1\_c): libC\\
inDebt(\allowbreak{}bob, book1\_a): pubA\\
inDebt(\allowbreak{}bob, book1\_c): libC\\
loans(\allowbreak{}alice, 0): libB\\
loans(\allowbreak{}alice, 0): pubA\\
loans(\allowbreak{}bob, 0): libB\\
loans(\allowbreak{}bob, 1): libC\\
loans(\allowbreak{}bob, 1): pubA\\
normalUser(\allowbreak{}alice): libB\\
normalUser(\allowbreak{}alice): libC\\
normalUser(\allowbreak{}alice): pubA\\
normalUser(\allowbreak{}bob): libB\\
normalUser(\allowbreak{}bob): pubA\\
\sout{borrowed(\allowbreak{}alice, book2\_c): libC}\\
\sout{loans(\allowbreak{}alice, 1): libC}\\
\end{minipage}
};
} % end node and chain
\draw (i7) -- (s7);

}
% \pause % uncomment here to animate

% start row=[5, 6, 7, 8, 9, 10], state=8
{{ [continue chain=trace]
\node[circle,draw,on chain=trace](i8) {$S_{8}$};
% label for s_7 -- s_8
\draw[-latex,thin](i7) -- % t=8
node[above]{\begin{tabular}{>{\centering}m{5cm}}
\\
\em return(\allowbreak{}alice, book2\_c): libC\\
 return(\allowbreak{}alice, book2\_c): libB\\
 return(\allowbreak{}alice, book2\_c): pubA\\
 viol(\allowbreak{}return(\allowbreak{}alice, book2\_c)): libB\\
 viol(\allowbreak{}return(\allowbreak{}alice, book2\_c)): pubA\\
 intReturn(\allowbreak{}alice, book2\_c): libC
\end{tabular}}
(i8);
}
{ [continue chain=state8 going below]
\node [on chain=state8,below=of i8,rectangle,draw,inner frame sep=0pt] (s8) {
% instant 8
\begin{minipage}{5cm}\raggedright\everypar={\hangindent=1em\hangafter=1}
\textbf{available(\allowbreak{}book2\_c): libC}\\
\textbf{allowance(\allowbreak{}alice, 3): pubA}\\
allowance(\allowbreak{}alice, 1): libB\\
allowance(\allowbreak{}alice, 2): libC\\
allowance(\allowbreak{}bob, 1): libC\\
allowance(\allowbreak{}bob, 2): libB\\
allowance(\allowbreak{}bob, 2): pubA\\
available(\allowbreak{}book1\_b): libB\\
available(\allowbreak{}book2\_a): pubA\\
available(\allowbreak{}book2\_b): libB\\
available(\allowbreak{}book3\_a): pubA\\
available(\allowbreak{}book4\_a): pubA\\
borrowed(\allowbreak{}bob, book1\_a): pubA\\
borrowed(\allowbreak{}bob, book1\_c): libC\\
inDebt(\allowbreak{}bob, book1\_a): pubA\\
inDebt(\allowbreak{}bob, book1\_c): libB\\
inDebt(\allowbreak{}bob, book1\_c): libC\\
loans(\allowbreak{}alice, 0): libB\\
loans(\allowbreak{}alice, 0): libC\\
loans(\allowbreak{}alice, 0): pubA\\
loans(\allowbreak{}bob, 0): libB\\
loans(\allowbreak{}bob, 1): libC\\
loans(\allowbreak{}bob, 1): pubA\\
normalUser(\allowbreak{}alice): libB\\
normalUser(\allowbreak{}alice): libC\\
normalUser(\allowbreak{}alice): pubA\\
normalUser(\allowbreak{}bob): libB\\
normalUser(\allowbreak{}bob): pubA\\
\sout{allowance(\allowbreak{}alice, 0): pubA}\\
\sout{allowance(\allowbreak{}alice, 1): pubA}\\
\sout{allowance(\allowbreak{}alice, 2): pubA}\\
\sout{allowance(\allowbreak{}alice, 3): pubA}\\
\sout{allowance(\allowbreak{}alice, 4): pubA}\\
\sout{allowance(\allowbreak{}alice, 5): pubA}\\
\end{minipage}
};
} % end node and chain
\draw (i8) -- (s8);

}
% \pause % uncomment here to animate

% start row=[5, 6, 7, 8, 9, 10], state=9
{{ [continue chain=trace]
\node[circle,draw,on chain=trace](i9) {$S_{9}$};
% label for s_8 -- s_9
\draw[-latex,thin](i8) -- % t=9
node[above]{\begin{tabular}{>{\centering}m{5cm}}
\\
\em extend(\allowbreak{}alice): pubA\\
 intExtend(\allowbreak{}alice): pubA
\end{tabular}}
(i9);
}
{ [continue chain=state9 going below]
\node [on chain=state9,below=of i9,rectangle,draw,inner frame sep=0pt] (s9) {
% instant 9
\begin{minipage}{5cm}\raggedright\everypar={\hangindent=1em\hangafter=1}
allowance(\allowbreak{}alice, 1): libB\\
allowance(\allowbreak{}alice, 2): libC\\
allowance(\allowbreak{}alice, 3): pubA\\
allowance(\allowbreak{}bob, 1): libC\\
allowance(\allowbreak{}bob, 2): libB\\
allowance(\allowbreak{}bob, 2): pubA\\
available(\allowbreak{}book1\_b): libB\\
available(\allowbreak{}book2\_a): pubA\\
available(\allowbreak{}book2\_b): libB\\
available(\allowbreak{}book2\_c): libC\\
available(\allowbreak{}book3\_a): pubA\\
available(\allowbreak{}book4\_a): pubA\\
borrowed(\allowbreak{}bob, book1\_a): pubA\\
borrowed(\allowbreak{}bob, book1\_c): libC\\
inDebt(\allowbreak{}bob, book1\_a): pubA\\
inDebt(\allowbreak{}bob, book1\_c): libB\\
inDebt(\allowbreak{}bob, book1\_c): libC\\
loans(\allowbreak{}alice, 0): libB\\
loans(\allowbreak{}alice, 0): libC\\
loans(\allowbreak{}alice, 0): pubA\\
loans(\allowbreak{}bob, 0): libB\\
loans(\allowbreak{}bob, 1): libC\\
loans(\allowbreak{}bob, 1): pubA\\
normalUser(\allowbreak{}alice): libB\\
normalUser(\allowbreak{}alice): libC\\
normalUser(\allowbreak{}alice): pubA\\
normalUser(\allowbreak{}bob): libB\\
normalUser(\allowbreak{}bob): pubA\\
\end{minipage}
};
} % end node and chain
\draw (i9) -- (s9);

}
% \pause % uncomment here to animate

% start row=[5, 6, 7, 8, 9, 10], state=10
{{ [continue chain=trace]
\node[circle,draw,on chain=trace](i10) {$S_{10}$};
% label for s_9 -- s_10
\draw[-latex,thin](i9) -- % t=10
node[above]{\begin{tabular}{>{\centering}m{5cm}}
\\
\em due(\allowbreak{}alice, book2\_c): libC
\end{tabular}}
(i10);
}
}
% \pause % uncomment here to animate

% fill nodes 11 to 10

% dummy arc to complete row
{ [continue chain=trace]
\node[on chain=trace] (i11) {};
\draw[color=white,-latex,dashed](i10) -- (i11);
}
% end row=[5, 6, 7, 8, 9, 10] of [[5, 6, 7, 8, 9, 10]]

\end{tikzpicture}
% close resizebox
}\\

\end{longtable}

